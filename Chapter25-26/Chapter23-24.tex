%========================%
%        Preamble        %
%========================%
\documentclass[12pt]{amsart}

    %========================%
%        Packages        %
%========================%

\usepackage[utf8]{inputenc}
%\usepackage{amsmath}    % Included in amsart package
%\usepackage{amsthm}     % 
\usepackage{amssymb}      % 
\usepackage{mathtools}      % Paired Limiter Macros
% \usepackage{mdframed}       % boxes for theorem
\usepackage{enumitem}     % Continuous numbering of lists
\usepackage[hidelinks]{hyperref}
\usepackage{tikz}
\usetikzlibrary{positioning}
\usepackage{blindtext}
\usepackage{graphicx}
\usepackage{float}

%========================% 
%          Title         %
%========================% 
\title{Chapters 25 and 26 Notes}
\author{Anish Sundaram}
\date{\today}

%========================% 
%        Theorems        %
%========================% 
\theoremstyle{definition}
\newtheorem{theorem}{Theorem}  % Boxed theorems
\newtheorem{definition}{Definition} % Definitions
\newtheorem{example}{Example}       %
\newtheorem{algorithm}{Algorithm}
\newtheorem*{proof*}{Proof}         % non-numbered
\newtheorem*{remark}{Remark}        %
\numberwithin{equation}{theorem}    % Local equation numbering

\setcounter{tocdepth}{3}      % Show subsubsections in contents

%========================% 
%        Macros          %
%========================% 
\DeclarePairedDelimiter\abs{\lvert}{\rvert}  % Vertical bars
\DeclarePairedDelimiter\norm{\lVert}{\rVert} % Double vertical bars
\newcommand{\drawvec}[1]{                    % matrices on one line
    \begin{bmatrix}
        #1
    \end{bmatrix}
}


% \begin{figure}[H]
%     \centering
%     \includegraphics[width=5in]{global-carbon-cycle.png}
%     \caption{The Global Carbon Cycle}
%     \label{global-carbon-cycle}
% \end{figure}

%========================% 
%         Document       %
%========================% 
\begin{document}

\maketitle

\tableofcontents

\section*{25 Current, Resistance, and Electromotive Force}

This chapter involves electric charges \textit{in motion} rather than static as before
An \textit{electric current} consists of charges in motion from one region to another. 
If the charges follow a conduct- ing path that forms a closed loop, the path is called an \textit{electric circuit.}


\subsection*{25.1 Current}
 \begin{definition}
    \textbf{Current ($I$)}:
    Any motion of charge from one region to another and is induced by the 
    ability of electrons to freely move. Current is zero everywhere in electrostatics.
    Current can be defined as $$I = \frac{dQ}{dt}$$ where Q is Coulombs and 
    its unit is Amperes(1 A = 1 C/s). Because Current is a scalar it must be 
    accompanied by a statement of direction: "25 Amps in the clockwise direction"
    
    \begin{remark}
        We define the current, denoted by $I$, to be in the direction 
    in which there is a flow of positive charge and describe currents as 
    though they consisted entirely of positive charge flow, even in cases 
    in which we know that the actual current is due to electrons.
    \end{remark}
 \end{definition}

 \begin{figure}[H]
    \centering
    \includegraphics[width=5in]{Media/Current.png}
    \caption{Current Flow Diagram}
    \label{Current Flow Diagram}
\end{figure}


 \begin{definition}
    \textbf{Drift Velocity ($v_d$)}:
    The average velocity of charged particles moving in the direction of the 
    electric force $\vec{F} = q\vec{E}$, though individual charge path is random. 
    This value gives us an alternate calculation of $I$ of 
    $$I = nqv_dA$$ where n is concentration of particles, q is unit charge and A is area.
 \end{definition}

 \subsubsection*{25.1.1 Direction of Current Flow}

\begin{remark}
    Different current-carrying materials may have differently charged moving particles
    In metals the moving charges are always electrons, while in an ionized gas (plasma) or an ionic 
    solution the moving charges may include both electrons and positively charged ions.
    In a semiconductor conduction is partly by electrons and partly by motion 
    of \textit{vacancies}, also known as holes; these are sites of missing electrons 
    and act like positive charges.
\end{remark}

\subsubsection*{25.1.2 Current Density}

\begin{definition}
    \textbf{Current Density (J)}:

   The current per unit cross-section area or $$J = I/A = nqv_d$$ where n is charge concentration, 
   q is the charge per particle and $v_d$ is the drift velocity.
\end{definition}

\subsection*{25.2 Resistivity}

\begin{theorem}
    \underline{Ohm's Law}:
    A relationship in an idealized model that states that the ratio of the electric
    field and the current density is constant in metals at a given temperature, and this is known
    as Resistivity.
\end{theorem}

\begin{definition}
    \textbf{Resistivity ($\rho$)}:
    The permitivity of electrons to move freely in a material, linked with resistance.
    Resistivity is defined as $$ \rho = \frac{E}{J}$$ where E is the magnitude 
    of the electric field and J is the current density. unit is ohm-meters(1 $\Omega \cdot m =V \cdot m/A$)
    Good insulators have high resistivity and conductors have low Resistivity.

    Resistivity can also be calculated as a function of Temperature:
    $$\rho(T) = \rho_0[1+\alpha(T-T_0)]$$ where $\alpha$ is a temperature 
    coefficient of resistivity and  $\rho_0$ being the Resistivity at a reference temperature $T_0$
\end{definition}

\begin{definition}
    \textbf{Conductivity}:
    The recipprocal of resistivity whose units are $(\Omega \cdot m)^{-1}$ Good conductors
    obviously have high Conductivity.
\end{definition}

\begin{definition}
    \textbf{Semiconductor}:
    Materials with propertis intermediate of metals and insulators, whose 
    resistivity is likewise between these two groups.
\end{definition}

\begin{remark}
    A material that obeys Ohm’s law reasonably well is called an \textit{ohmic 
    conductor} or a \textit{linear conductor}, those that dont are \textit{nonohmic}, 
    or \textit{nonlinear}. In the latter materials, J depends on E in a more complicated manner.
\end{remark}

\begin{figure}[H]
    \centering
    \includegraphics[width=5in]{Media/Resistivity.png}
    \caption{Resistivity Across Material Types}
    \label{Resistivity Across Material Types}
\end{figure}

\subsection*{25.3 Resistance}

\begin{definition}
    \textbf{Resistance ($R$)}:
    The ratio of Potential Difference V to Current I for a particular conductor.
    Resistance measures the opposition to current flow in an electrical circuit.
    The resistance of a conductor can be calculated by the equation
    $$R = \frac{\rho L}{A} = \frac{V}{I}$$ and its unit is Ohm (1 $\Omega = 1V/A$)
    and similar to resistivity can be ade a function of Temperature:
    $$R(T) = R_0[1+\alpha(T-T_0)]$$
\end{definition}

\begin{figure}[H]
    \centering
    \includegraphics[width=5in]{Media/Resistor.png}
    \caption{Labeling Guide for Resistors}
    \label{Labeling Guide for Resistors}
\end{figure}

\subsection*{25.4 Electromotive Force and Circuits}

\begin{definition}
    \textbf{Electromotive Force ($\mathcal{E}$)}:
    In an electric circuit there must be a device somewhere in the loop that 
    acts like the water pump in a water fountain. This pumping action is called 
    Electromotive Force or emf and the device is called a source of emf. This is not actually
    a force but a energy-per-unit-charge and thus has the unit Volt (1 V = 1 J/C).

    Emf can be calculated in multiple ways:
    $$\mathcal{E} = V_{ab} = IR $$ for ideal sources and where $V_{ab}$ is the Terminal Voltage
    or $$\mathcal{E} = V_{ab} + Ir$$ for when there is internal resistance (r).
\end{definition}

\begin{figure}[H]
    \centering
    \includegraphics[width=5in]{Media/Circuit.png}
    \caption{Table of Circuit Labels}
    \label{Table of Circuit Labels}
\end{figure}

\end{document}