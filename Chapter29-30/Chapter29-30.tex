%========================%
%        Preamble        %
%========================%
\documentclass[12pt]{amsart}

    %========================%
%        Packages        %
%========================%

\usepackage[utf8]{inputenc}
%\usepackage{amsmath}    % Included in amsart package
%\usepackage{amsthm}     % 
\usepackage{amssymb}      % 
\usepackage{mathtools}      % Paired Limiter Macros
% \usepackage{mdframed}       % boxes for theorem
\usepackage{enumitem}     % Continuous numbering of lists
\usepackage[hidelinks]{hyperref}
\usepackage{tikz}
\usetikzlibrary{positioning}
\usepackage{blindtext}
\usepackage{graphicx}
\usepackage{float}

%========================% 
%          Title         %
%========================% 
\title{Chapters 29 and 30 Notes}
\author{Anish Sundaram}
\date{\today}

%========================% 
%        Theorems        %
%========================% 
\theoremstyle{definition}
\newtheorem{theorem}{Theorem}  % Boxed theorems
\newtheorem{definition}{Definition} % Definitions
\newtheorem{example}{Example}       %
\newtheorem{algorithm}{Algorithm}
\newtheorem*{proof*}{Proof}         % non-numbered
\newtheorem*{remark}{Remark}        %
\numberwithin{equation}{theorem}    % Local equation numbering

\setcounter{tocdepth}{3}      % Show subsubsections in contents

%========================% 
%        Macros          %
%========================% 
\DeclarePairedDelimiter\abs{\lvert}{\rvert}  % Vertical bars
\DeclarePairedDelimiter\norm{\lVert}{\rVert} % Double vertical bars
\newcommand{\drawvec}[1]{                    % matrices on one line
    \begin{bmatrix}
        #1
    \end{bmatrix}
}


% \begin{figure}[H]
%     \centering
%     \includegraphics[width=5in]{global-carbon-cycle.png}
%     \caption{The Global Carbon Cycle}
%     \label{global-carbon-cycle}
% \end{figure}

%========================% 
%         Document       %
%========================% 
\begin{document}
\maketitle

\tableofcontents

\section*{29 Electromagnetic Induction}

If the magnetic flux through a circuit changes, an emf and a current are induced in the circuit. In a power-generating station, magnets move relative to coils of wire to produce a changing magnetic flux in the coils and hence an emf.
The central principle of electromagnetic induction is Faraday’s law. This law relates induced emf to changing magnetic flux in any loop, including a closed circuit. We also discuss Lenz’s law, which helps us to predict the directions of induced emfs and currents. These principles will allow us to under- stand electrical energy-conversion devices such as motors, generators, and transformers.


\subsection*{29.1 Induction Experiments}

\begin{definition}
    \textbf{Induced Current}:
    Inducing a current to flow by moving a magnet into a coil or by moving a coil
    in a magnetic field causing flux. This flowing Currentresults in an \textbf{Induced emf}
\end{definition}

\subsection*{29.2 Faraday's Law}

\begin{definition}
    \textbf{Faraday’s Law}:
    The common element in induction effects is changin magnetic flux through a circuit
    where magnetic flux can be found as $\Phi_B = \int B dAcos\phi$ where $\phi$ is the angle between the face and the magnetic field. From this we can derive an equation for the induced emf: 
    $$\mathcal{E} = -\frac{d\Phi_B}{dt}$$ or put into words that the induced emf equals the negative time rate of magnetic flux.
    \begin{remark}
        This relationship is valid whether the flux change is caused by a changing magnetic field, motion of the loop, or both.
    \end{remark}
    \begin{remark}
    If a coil has N identical turns and if the flux varies at the same rate through each turn, the total rate of change through all turns is N times that for a single turn. In other words $$\mathcal{E} = -N\frac{d\Phi_B}{dt} = \mu_0NA\frac{dl}{dt}/L$$
    \end{remark}
\end{definition}

\begin{definition}
    \textbf{Finding Direction of Induced emf}:
    To find the direction of an induced emf or current we can us 4 rules:
    \begin{enumerate}
        \item Define a positive direction for the vector area $\vec{A}$
        \item From the directions of $\vec{A}$ and the magnetic field $\vec{B}$ , determine the sign of
        the magnetic flux $\Phi_B$ and its rate of change $d\Phi_b/dt$
        \item Determine the sign of the induced emf or current. If the flux is increasing, so $d\Phi_b/dt$ is positive, then the induced emf or current is negative; if the flux is decreasing, $d\Phi_b/dt$ is negative then the infuced emf or current is positive. 
        \item Finally, use your right hand to determine the direction of the induced emf or current. Curl the fingers of your right hand around the $\vec{A}$ vector, with your right thumb in the direction of $\vec{A}$ . If the induced emf or current in the circuit is \textit{positive}, it is in the same direction as your curled fingers; if the induced emf or current is \textit{negative}, it is in the opposite direction
    \end{enumerate}
\end{definition}

\subsection*{29.3 Lenz's Law}

\begin{definition}
    \textbf{Lenz’s Law}:
    The direction of any magnetic induction effect is such as to oppose the cause of the effect.Lenz’s law gives only the direction of an induced current; the magnitude of the current depends on the resistance of the circuit.
\end{definition}
\subsection*{29.4 Motional Electromotive Force}

\begin{definition}
    \textbf{Motional Electromotive Force}:
    The emf generated by a moving electric conductor in the presence of a magnetic field. This emf can be calculated by the equation $$\mathcal{E} = vBL$$ where $v$ is the conductor speed and $L$ is the length of the conductor.
    This equation is generalized as a path integral where the velocity is perpendicular to the magnetic field and the resulting vector is parallel to the length:
    $$\mathcal{E} = \oint(\vec{v} \times \vec{B} \cdot d\vec{l})$$ This is function al for all eklements of a closed conducting loop.
\end{definition}

\subsection*{29.5 Induced Electric Fields}
\begin{definition}
    \textbf{Induced Electric Fields/Nonelectrostatic Fields}:
    When an emf is induced by a changing magnetic flux through a stationary conductor, there is an induced electric field of nonelectrostatic origin which is nonconservative and cannot be associated with potential. This relationship can be described as: 
    $$\oint \vec{E} \cdot d\vec{l} = \mathcal{E} = -\frac{d\Phi_B}{dt}$$
    For a loop we get the equation $$E = \frac{1}{2\pi r}{|\frac{d\Phi_B}{dt}|}$$

    \begin{remark}
        The induced electric field is nonconservative because it does net work in moving a charge over a closed path, whereas the electrostatic field is conservative and does no net work over a closed path. Hence, electric potential can be associated with the electrostatic field, but not with the induced field.
    \end{remark}
\end{definition}

\subsection*{Eddy Currents}

\begin{definition}
    \textbf{Eddy Currents}:
    Induced currents caused by metals moving in magnetic fields that circulate thorughout the volume of a material, these are often swirling in shape. 
\end{definition}

\subsection*{29.7 Displacement Current and Maxwell's Equations}

\begin{definition}
    \textbf{Displacement Current}:
    Imaginary current made to fit Maxwell's Equations
    $$I_d = \frac{\epsilon_0 \Delta \Phi_E}{\Delta t}$$
\end{definition}

\begin{definition}
    \textbf{Maxwell's Equations}:
    The Complete versions of previous equations
    \begin{enumerate}
        \item Gauss's Law for $\vec{E}$: $$\oint \vec{E} \cdot d\vec{A} = \frac{Q_{encl}}{\epsilon_0}$$ 
        \item Gauss's Law for $\vec{B}$: $$\oint \vec{B} \cdot d\vec{A} = 0$$
        \item Farada's Law for stationary integration path:$$\oint \vec{E} \cdot d\vec{l} = -\frac{d\Phi_B}{dt}$$  
        \item Ampere’s law for a stationary integration path:
        $$\oint \vec{B} \cdot d\vec{l} = \mu_0 (i_c + \epsilon_0\frac{d\Phi_E}{dt})encl$$ where $i_c$ is the conduction current through the path.
    \end{enumerate}
\end{definition}

\section*{30 Inductance}
A changing current in a coil induces an emf in an adjacent coil. The coupling between the coils is described by their \textit{mutual inductance}. A changing current in a coil also induces an emf in that same coil. Such a coil is called an \textit{inductor}, and the relationship of current to emf is described by the \textit{inductance} (also called self- inductance) of the coil. If a coil is initially carrying a current, energy is released when the current decreases; this principle is used in automotive ignition systems.

\subsection*{30.1 Mutual Inductance}

\begin{definition}
    \textbf{Mutual Inductance ($M$)}:
    When an emf is produced in a coil because of the change in current in a coupled coil , the effect is called mutual inductance. Can be described as:
    \begin{center}
        $\mathcal{E}_2 = -M\frac{di_1}{dt}$ and $\mathcal{E}_1 = -M\frac{di_2}{dt}$ 
    \end{center}
    where each emf is the negative of the rate of change of current in the other coil times the mutual inductance. 

    M itself can be described using lenz's law as $$M = \frac{N_2 \Phi_{b2}}{i_1} = \frac{N_1\Phi_{B1}}{i_2}$$ 

    The unit of mutual inductance is called the henry 
    \begin{center}
        (1$H$ = 1$Wb/A$ = 1$V\cdot s/A$ = 1$\Omega \cdot s$ = 1$J/A^2$)
    \end{center}
    
\end{definition}

\subsection*{30.2 Self-Inductance and Inductors}

\begin{definition}
    \textbf{Self-induced emf}:
    Any circuit that carries a varying current has an emf induced in it by the variation in its own magnetic field. Such an emf is called a self-induced emf.
\end{definition}

\begin{definition}
    \textbf{Inductance/Self-Inductance($L$)}:
    Self-induced emfs can occur in any circuit so long as there is variable current. The effect of self-inductance is magnified if the circuit includes a coil. For this there is the relation:
    $$L = \frac{N\Phi_B}{i}$$ where $i$ is the current in the coil and $\Phi_B$ is the flux due to the current through each turn of the coil.
    \begin{remark}
        This relationship can be conbined with Faraday's law resulting in:
        $$\mathcal{E} = -L\frac{di}{dt}$$ where $L$ is the inductance of the circuit.
    \end{remark}
\end{definition}

\begin{definition}
    \textbf{Inductor/Choke}:
    Circuit device that is designed to have a particular inductance.
\end{definition}

\subsection*{30.3 Magnetic-Field Energy}

\begin{definition}
    \textbf{Total Energy in Inductor}:
    We can find the change in energy as the change in Power.
    $$U = L\int^I_0idt = .5LI^2$$ where I is final current and i is the instantaneous current.
    \begin{remark}
        This equation can be expanded for toiroidal solenoids where $L = \frac{\mu_0N^2A}{2\pi r}$ which results in $$U = .5LI^2 = .5\frac{\mu_0N^2A}{2\pi r}I^2$$
    \end{remark}
\end{definition}

\begin{definition}
    \textbf{Magnetic Energy Density($u$)}:
    The magnetic analog of energy per unit volume in an electric field in vacuum,$u=.5\epsilon_0E^2$. For Magnetism the equation is: 
    $$u = \frac{B^2}{2\mu_0}$$
    \begin{remark}
        If the material inside the toroid is not vacuum but a material we can use its magnetic permeability$\mu = K_m\mu_0$ and replace the above equation with $$U = \frac{B^2}{2\mu}$$
    \end{remark} 
\end{definition}

\subsection*{30.4 the R-L Circuit}

\begin{definition}
    \textbf{R-L Circuit}:
    A circuit that includes both a resistor and an inductor, and possibly a source of emf.The resistor R may be a separate circuit element, or it may be the resistance of the inductor windings.
    \begin{remark}
        For an R-L circuit the time constant $\tau = \frac{L}{R}$ where current increases and drops exponentially. 
    \end{remark}
\end{definition}

\subsection*{30.5 The L-C Circuit}

\begin{definition}
    \textbf{L-C Circuit}:
    A circuit that contains inductance L and capacitance C undergoes electrical oscillations with an angular frequency $\omega$ that depends on L and C.
\end{definition}

\begin{definition}
    \textbf{Angular Frquency of Oscillation in an L-C circuit}:
    $$\omega = \sqrt{\frac{1}{LC}}$$ 
    \begin{remark}
        This is analogous to a mechanical harmonic oscillator, with inductance L analogous to mass m, the reciprocal of capacitance 1/C to force constant k, charge q to displacementx, and current i to velocity $v_x$
    \end{remark}
\end{definition}


\subsection*{30.6 L-R-C Circuits}

\begin{definition}
    \textbf{L-R-C Circuit}:
    A circuit that contains inductance, resistance, and capacitance undergoes damped oscillations for sufficiently small resistance.
\end{definition}


\begin{definition}
    \textbf{Angular Frquency of Oscillation in underdamped L-R-CC circuits}:
    For underdamped oscillations in an L-R-C circuits the angular frequency can be described as $$\omega' = \sqrt{\frac{1}{LC} - \frac{R^2}{4L^2}}$$

    \begin{remark}
        The frequency $\omega'$ of damped oscillations depends on the values of L, R, and C. As R increases, the damping increases; if R is greater than a certain value, the behavior becomes overdamped and no longer oscillates. 
    \end{remark}
\end{definition}
\end{document}
